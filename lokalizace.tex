\documentclass[12pt,a4paper]{report}

% Babel, písmo, kódování 
\usepackage[czech]{babel}
\usepackage[utf8]{inputenc}
\usepackage{lmodern}
\usepackage[T1]{fontenc}
\usepackage[babel=true]{microtype}

\begin{document}
\section*{Lokalizace softwaru}

Lokalizace softwaru je činnost, při které se řetězce, které jsou zobrazovány v
různých hlášeních, výstupech a v uživatelském rozhraní daného softwaru,
překládají do jiných jazyků tak, aby s ním mohli pracovat i lidé, kteří
původnímu jazyku nerozumějí. Lokalizace velmi úzce souvisí s pojmem
internacionalizace.  Tento pojem označuje činnost, při které jsou  zdrojové
kódy programů psány tak, aby bez dalších úprav správně pracovaly v různých
národních prostředích se správnými lokalizovanými řetězci a se správnými
národními zvyky, jako je například způsob psaní času, data, měny, desetinné
čárky a další. Pro obě činnosti se někdy používá souhrnný termín globalizace.
Všechny tři termíny se běžně zkracují nahrazením prostředních písmen v
anglickém názvu jejich počtem: l10n, i18n a g11n.

Internacionalizace, pokud se na ní myslí již v návrhu programu a pokud je
správně implementována, nestojí mnoho peněz a nedělá tolik problémů. To je dáno
zejména tím, že internacionalizace se provádí pouze jednou (bez ohledu na počet
jazyků, do kterých se bude lokalizovat) a v dnešní době existuje velká podpora
v systémových knihovnách a ve standardních knihovnách většiny jazyků. (O to
více práce ale mají s internacionalizací výrobci operačních systémů).

Naproti tomu lokalizace bývá velmi drahá a problémová. Její cena je téměř přímo
úměrná počtu jazyků, do kterých chceme software lokalizovat, a množství řetězců
v daném softwaru. Většina nákladů na lokalizaci se spolkne samotný překlad,
který většinou zajišťují externí dodavatelé. Velmi mnoho peněz a úsilí
stojí také správa databází, ve kterých jsou lokalizované řetězce uloženy, testování
lokalizovaného softwaru a oprava nalezených chyb. Jednou z mála činností s
konstantními náklady v rámci lokalizace, je implementace a udržování
lokalizačních nástrojů. 

Mnoho softwarových firem, které chtějí dodávat svůj software globálně, proto
řeší otázku, co všechno se vyplatí lokalizovat. V některých oblastech totiž
není lokalizace tak důležitá a někdy je dokonce nežádoucí. Uživatelský software
jako je například kancelářský balík Microsoft Office, se překládá témeř vždy.
Systémové nástroje, se kterými pracují výhradně odborníci se však překládat
nemusí a podle názoru některých se takový software ani překládat nemá.
Předpokládá se totiž, že odborníci umějí dobře anglicky a je tedy lepší, když
software zůstane v angličtině a používají se ustálené anglické odborné termíny.
Jako programátor používám sám raději nelokalizovaný software v angličtině. Mimo
jiné mi to umožňuje hledat chybové hlášky na diskusních fórech. I
nelokalizovaný software by však měl být vždy internacionalizován, aby správně
zobrazoval data, čas, měnu, ale aby také podporoval různé kódování textu, nebo
alespoň univerzální kódování utf-8.

Vliv na to, co se bude lokalizovat a co ne, má zejména velikost trhu, používané
písmo ale také tradice. Například v brazilském trhu není tolik obvyklé, aby
programátoři a administrátoři uměli dobře anglicky.  Proto se zde mnohem více
vyžaduje lokalizace softwaru a do brazilské portugalštiny se lokalizuje
častěji, než do evropských jazyků, i když je brazilský trh menší, než ten
evropský. Dokladem toho je i například zprovoznění brazilské mutace znalostního
serveru Stackoverflow.com, který jinak, jak sami provozovatelé zdůrazňují,
nechtějí provozovat v jiných jazycích, aby se znalosti a zkušenosti soustředily
na jednom místě.

Ve firmě Oracle pracuji v týmu Solaris Globalization, který má na starosti
internacionalizaci a částečně i lokalizaci operačního systému Solaris.  Mnoho
problémů s lokalizací a internacionalizací zmíněných na přednášce tak mohu
jenom potvrdit a kdyby zde bylo více prostoru, tak bych přidal další problémy z
vlastní zkušennosti.

Většina operačního systému Solaris se překládá pouze do několika základních
jazyků (zjednodušená a tradiční čínština, japonština, korejština,
francouzština, němčina, španělština, italština a brazilská portugalština).
Některé části systému (zejména grafické prostředí a programy) se přebírají z
komunitního vývoje se všemi dostupnými překlady. Některé části systému, jako je
například dokumentace se naopak nepřekládají ani do všech základních jazyků.
Abych se oklikou vrátil zpět k přednášce, velmi mě překvapilo, že se operační
systém firmy IBM, což je vlastně přímý konkurent systému Solaris, lokalizuje do
češtiny. Český trh je velmi malý a administrátoři jsou zde zvyklí pracovat s
nelokalizovaným softwarem.

\end{document}
