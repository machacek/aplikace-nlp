\documentclass[12pt,a4paper]{report}

% Babel, písmo, kódování 
\usepackage[czech]{babel}
\usepackage[utf8]{inputenc}
\usepackage{lmodern}
\usepackage[T1]{fontenc}
\usepackage[babel=true]{microtype}

\begin{document}
\section*{Lokalizace softwaru}

Lokalizace softwaru je činnost, při které se řetězce, které jsou zobrazovány v
různých hlášeních, výstupech a v uživatelském rozhraní daného softwaru,
překládají do jiných jazyků tak, aby s ním mohli pracovat i lidé, kteří
původnímu jazyku nerozumějí. Lokalizace velmi úzce souvisí s pojmem
internacionalizace.  Tento pojem označuje činnost, při které jsou  zdrojové
kódy programů psány tak, aby bez dalších úprav správně pracovaly v různých
národních prostředích se správnými lokalizovanými řetězci a se správnými
národními zvyky, jako je způsob psaní času, data, měny, desetinné čárky a
dalších. Pro obě činnosti se někdy používá termín globalizace. Všechny tři
termíny se běžně zkracují nahrazením prostředních písmen v anglickém názvu
jejich počtem: l10n, i18n a g11n.

Internacionalizace, pokud se na ní myslí již v návrhu programu a pokud je
správně implementována, nestojí mnoho peněz a nedělá tolik problémů. To je dáno
zejména tím, že internacionalizace se provádí pouze jednou (bez ohledu na počet
jazyků, do kterých se bude lokalizovat) a v dnešní době existuje velká podpora
v systémových knihovnách a ve standardních knihovnách většiny jazyků. (O to více práce ale mají
s internacionalizací výrobci operačních systémů).

Naproti tomu lokalizace bývá velmi drahá a problémová. Její cena je téměř přímo
úměrná počtu jazyků, do kterých chceme software lokalizovat. Většina nákladů na
lokalizaci se spolkne samotný překlad, který většinou zajišťují externí
dodavatelé.  Ale velmi mnoho peněz a úsilí stojí správa databází, ve kterých
jsou lokalizované řetězce uloženy, testování lokalizovaného softwaru a oprava
nalezených chyb. Jednou z mála činností s konstantními náklady v rámci
lokalizace, je implementace a udržování lokalizačních nástrojů. 

Mnoho softwarových firem, které chtějí dodávat svůj software globálně, proto
řeší otázku, co všechno se vyplatí lokalizovat. V některých oblastech totiž
není lokalizace tak důležitá a někdy je dokonce nežádoucí. Uživatelský software
jako je například kancelářský balík Microsoft Office, se překládá témeř vždy. Systémové
nástroje, se kterými pracují výhradně odborníci se však překládat nemusí a podle názoru některých se takový 
software ani překládat nemá. Předpokládá se totiž, že odborníci umějí dobře anglicky a je tedy lepší, 
když software zůstane v angličtině.

Ve firmě Oracle pracuji v týmu Solaris Globalization, který má na starosti internacionalizaci a částečně i lokalizaci 
operačního systému Solaris. Z velkou částí problémů zmiňovaných na přednášce jsem se setkal. 





% prekladat pouze do nekterych jazyku, prekladat ma cenu zejmena uzivatelske
% rozhranni pro bezne uzivatele, pokud prekladat  i command line interface, tak
% pouze pro core jazyky. 

% vlastni zkusennost, neni to jednoduche, je to drahe, vyuzivaji se translation
% memory, cena za jednu vetu se urcuje podle podobnosti k nejblizsi bete v
% prekladove pameti.  

% prekladani dokumentace, nejcastejsi problemy, preklad neni synchronizovan s
% anglickou verzi, overtranslation, testovani dokumentace,

% bug s prejmenovanim Thajwanu



\end{document}
