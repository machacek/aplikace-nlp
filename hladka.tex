\documentclass[12pt,a4paper]{report}

% Babel, písmo, kódování 
\usepackage[czech]{babel}
\usepackage[utf8]{inputenc}
\usepackage{lmodern}
\usepackage[T1]{fontenc}
\usepackage[babel=true]{microtype}

\begin{document}
\section*{Čapek}

-gamifikace, crowdsourcing, dobry napad, resi dve veci, potrebuje lepe dotahnout anotacni
prostredi, i kdyby fungovalo dobre, je stale moc slozite, pro skolaky casto slozite vety.  
Mozna by to chtelo nechat naprogramovat profesionalne, aby nezapadla dorbra myslenka. Chtelo by to nejake
intuitivnejsi a prehlednejsi  ovladani, napriklad vytvoreni zavislosti tazenim z jednoho uzlu do druheho,
to hraje velmi dulezitou roli a velmi to ovlivni celkovy dojem. 

\section*{Rozpoznávání L1 jazyka}

Paní doktorka Hladká řešila tuto úlohu v rámci soutěže Native Language
Identification (NLI) Shared Task 2013.  Cílem této úlohy je rozpoznat mateřský
jazyk autora anglicky napsané eseje.  Problém se tedy dá formulovat jako úloha
klasifikace strojového učení a jak se dalo očekávat, všichni účastníci soutěže
to také řešili pomocí technik strojového učení. 

Myslím, že je to velmi zajímavý problém z nejméně dvou pohledů. Z pohledu inženýra 
strojového učení



zajimava uloha pro lingvisty, otazka, zda to ma nejake prakticke vyuziti

otazka, zda by expert dokazal ulohu resit

-pouze feature engineering, cekal bych nejake zajimavejsi rysy nebo originalni zpusoby reseni. 

napad: na jednotlivych tridach natrenovat language modely, pomoci nich pak spocitat pravdepodobnost textu. Nebo aplikovat
model zasumeneho signalu, tedy myslet si, ze autor napsatl esej v jazyce L1, pak se ale esej v zasumenem kanalu prelozila na jazyk L2.





\end{document}
