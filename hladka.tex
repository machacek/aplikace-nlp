\documentclass[12pt,a4paper]{report}

% Babel, písmo, kódování 
\usepackage[czech]{babel}
\usepackage[utf8]{inputenc}
\usepackage{lmodern}
\usepackage[T1]{fontenc}
\usepackage[babel=true]{microtype}

\begin{document}

\section*{Rozpoznávání L1 jazyka}

Paní doktorka Hladká řešila tuto úlohu v rámci soutěže Native Language
Identification (NLI) Shared Task 2013.  Cílem této úlohy je rozpoznat mateřský
jazyk autora anglicky napsané eseje.  Myslím, že je to velmi zajímavý problém
pro inženýry strojového učení i pro lingvisty, ale nevidím v tom žádné
praktické využití. 

Problém se tedy dá formulovat jako úloha klasifikace strojového učení a jak se
dalo očekávat, všichni účastníci soutěže to pomocí technik strojového učení
řešili. K problému se dá však přistoupit dvěma způsoby. 

První způsob je čisté strojové učení. Vygeneruje se co nejvíce rysů, z těch se
poté vyberou užitečné rysy a pak už se jen zkouší všechny možné metody
strojového učení a vybere se ta, která dává na dev-testu nejlepší výsledky.
Úloha je jistě zajímavá i z tohoto čistě inženýrského pohledu: k dispozici je
málo trénovacích pozorování, rysů je hodně a jazyků, mezi kterými klasifikátor
vybírá je také hodně. Přitom je to úloha, která je velmi složitá (není-li
nemožná) na řešení i pro člověka. 

Jako druhý způsob řešení vidím to, že se nejdříve využijí nějaké poznatky z
lingvistiky nebo psychologie. Je potřeba najít souvislosti mezi používáním
mateřského jazyka (L1) a jazyka, který se učíme jako druhý (L2). Jinými slovy,
lingvista by měl zjistit, jakým způsobem ovlivňuje mateřský jazyk používání
druhého jazyka a na základě toho extrahovat signifikantní rysy. Zbytek už je
pochopitelně opět strojové učení podobné jako v prvním případě. 

Z prezentace paní doktorky Hladké se mi zdálo, že její kolektiv přistoupil k
problému pouze prvním způsobem a nehledal nějaké hlubší souvislosti. Na její
obranu je však potřeba říci, že to podobným způsobem řešili i ostatní
soutěžící. 

Závěrem dodám svůj návrh na řešení problému, který by asi z různých důvodů
nefungoval, ale napadl mě jako první. Na anglických textech neanglických
rodilých mluvčích by se natrénovaly jazykové modely, pro každý L1 jazyk
samostatný model.  Pro klasifikovanou esej by se poté spočítala pravděpodobnost
vygenerování každým z natrénovaných jazykových modelů a jako L1 jazyk by se
vybral ten s nejvyšší pravděpodobností. Toto řešení by však vyžadovalo velmi
mnoho trénovacích dat. 

\section*{Čapek}

Program Čapek řeší dvě věci, je to pomůcka pro žáky a učitele na procvičování
větných rozborů a zároveň tento nástroj slouží pro získávání nových anotací do
závislostního korpusu. Je to výborný nápad, který využívá fenoménu zvaného
crowdsourcing. Mám k tomu dvě poznámky.

S odstupem času mi není jasné, zda bude anotované věty někdo po žácích
kontrolovat. Asi by mělo smysl, kdyby to kontroloval učitel, ale vzhledem k
tomu, že každý žák anotuje jiné věty, to bude pro učitele velmi náročné. 

Další poznámka se týká anotačního prostředí. Úspěšnost celého projektu závisí
na komfortnosti anotační aplikace.  Ta musí být jednoduchá a přehledná, její
ovládání musí být intuitivní. Žáci i učitelé musí hned vědět, jak v aplikaci
vytvářet větné rozbory. Z aplikace, kterou jsme testovali na cvičení, však
takový pocit nemám. Zásadním nedostatkem aplikace byla nemožnost kroku zpět,
zejména při spojování uzlů. Toto je zvlášť nepříjemné, když se závislost dvou
uzlů vytváří podobným způsobem jako, když se dva uzly spojují. Velmi často se
mi stávalo, že jsem dva uzly spojil, místo toho abych mezi nimi vytvořil
závislost. Jednu větu jsem kvůli tomu musel anotovat asi čtyřikrát. I bez
tohoto nedostatku se mi však aplikace zdá stále neintuitivní. Dokážu si
například představit, že se závislosti budou vytvářet tažení z jednoho uzlu do
druhého. Vše by mělo být doplněno názornou vizuální (možná i zvukovou) odezvou,
aby si uživatel byl jistý, že děla to, co chce. Aplikace by v ideálním případě
měla fungovat stejně ve webovém prohlížeči i na tabletu. Stálo by za zvážení,
zda by se taková aplikace neměla nechat vytvořit buď profesionálně externí
firmou, nebo v rámci většího softwarového projektu s důrazem na uživatelskou
přívětivost a moderní HTML technologie. 

\end{document}
