\documentclass[12pt,a4paper]{report}

% Babel, písmo, kódování 
\usepackage[czech]{babel}
\usepackage[utf8]{inputenc}
\usepackage{lmodern}
\usepackage[T1]{fontenc}
\usepackage[babel=true]{microtype}

\begin{document}

\section*{Watson}

Watson (nazvaný podle zakladatele firmy IBM) je počítačový systém schopný
odpovídat na otázky položené v lidské řeči a odpovědi generovat opět v lidské
řeči. Jeho doména není omezená. Systém využívá mnoho různých technik počítačové
lingvistiky, mezi jinými to jsou analýza mluvené řeči, porozumění otázky,
vyhledávání informací, generování, ohodnocení a spojování hypotéz, hledání a
ohodnocení dokladů pro odpověď, generování přirozené řeči a syntéza hlasu.

Po vítězství počítače Deep Blue v šachu nad Garrym Kasparovem, se chtěla firma
IBM opět vytáhnout a dokázat světu svůj technologický náskok. Začala tedy
vyvíjet systém Watson, který v roce 2010 porazil nejlepší hráče ve hře Jeopardy
(Americká obdoba hry Riskuj). Vítězství v této hře bylo především marketingový
tah, firma plánuje používat systém Watson při rozhodování ve zdravotnictví a
při investování.

Dle mého názoru se jedná opět o velmi zajímavý projekt, který spojuje mnoho
různých disciplín matematické lingvistiky. IBM dokazuje, že umí využít
špičkové vědce, aby vytvořila komerčně úspěšný produkt. 

Mám obecnou představu, jak funguje systém Watson ve hře Jeopardy, zajímalo by
mě ale, jakým způsobem chce firma IBM využít systém například ve zdravotnictví.
Například si nedokážu představit jak by konkrétně doktoři se systémem
pracovali. Budou se doktoři ptát jaká nemoc může způsobovat nalezené příznaky?
Bohužel mě nenapadlo zeptat se na to během prezentace. 

\section*{Inteligentní zadávání textu}

Ve druhé části prezentace firmy IBM nám byl představen projekt, na kterém
pražská pobočka firmy pracuje v posledních letech. Jedná se o systém zadávání
textu (například krátkých textových zpráv) během řízení auta. Líbilo se mi, že
na to šli systematicky a navrhli různé konfigurace, které pak testovali.
Nejdůležitějším kriteriem při testování byla pozornost řidiče během řízení,
kterou měřili pomocí chytře vymyšleného simulátoru. 

Při praktické ukázce jsme si vyzkoušeli prototyp jejich systému a musím říct,
že jsem byl velmi překvapen kvalitou sytému a jeho použitelností. Systém
dokázal napoprvé správně napsat téměř vše a pokud ne, dalo se to opravit
kouzelnou formulkou ,,Řekl jsem ...''. Dokážu si dobře představit, že bych
takový systém v autě využil a že bych se časem dokázal naučit používat
ho ještě efektivněji. Při praktické ukázce jsme zkoušeli konfiguraci s
displejem. Nedokážu posoudit jak moc displej snižuje pozornost řidiče, ale
věřím, že s~moderními head up displeji to nebude mít velký vliv na bezpečnost.



\end{document}
