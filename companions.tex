\documentclass[12pt,a4paper]{report}

% Babel, písmo, kódování 
\usepackage[czech]{babel}
\usepackage[utf8]{inputenc}
\usepackage{lmodern}
\usepackage[T1]{fontenc}
\usepackage[babel=true]{microtype}

\begin{document}
\section*{Projekt Companions}

Cílem projektu Companions bylo vytvořit virtuální společnici, která si bude
povídat se seniory a zabaví je tak. Je to dobrý nápad a věřím, že dialogové
systémy tohoto typu mají velký smysl. V tomto případě byla navíc doména omezena
na rozhovory týkající se rodinných fotek. Toto omezení bylo použito
pravděpodobně proto, aby se snížilo potřebné množství dat pro trénink modelů a
aby se zmenšila velikost sítě dialogu. Toto omezení je tedy pochopitelné a
věřím, že s větším množstvím prostředků by šlo vyvinout univerzálnější systém,
který bude dosahovat podobné kvality.

Abych mohl zhodnotit úspěch nebo neúspěch projektu, musel bych znát skutečné
cíle tohoto projektu. Pokud šlo o aplikovaný výzkum a skutečným cílem bylo
vytvořit hotový a odladěný produkt, který se bude používat k zabavení seniorů,
tak projekt skončil neúspšěchem. Přeci jenom do použitelného produktu schází
mnoho.

Pokud šlo o základní výzkum, tak věřím, že projekt dopadl, zvláště po zmíněných 
problémech s financováním, velmi dobře. Zdá se mi, že řešitelé projektu 
na to šli spr


asi spise neuspech, ale takovy projekt ma velky smysl do budoucna. Verim, ze az budu
senior, tak se takovi spolecnici budou bezne pouzivat.

pri ukazce velmi zlobilo porozumneni reci, pravdepodone spatne nakonfigurovane, tezko posoudit.
V porovnani s rozpoznavanim reci od IBM to bylo velmi slabe. 

neni to nejak nemoralni, se seniory bychom meli travit vice casu my sami, na
druhou stranu takovy virtualni spolecnik nam v tom nebrani a muze seniory
zabavit v dobe, kdy jsou sami

aplikovany nebo zakladni vyzkum? Ukazka spatneho rozhazovani evropskych penez. 


\end{document}
