\documentclass[12pt,a4paper]{report}

% Babel, písmo, kódování 
\usepackage[czech]{babel}
\usepackage[utf8]{inputenc}
\usepackage{lmodern}
\usepackage[T1]{fontenc}
\usepackage[babel=true]{microtype}

\begin{document}
\section*{Projekt Companions}

Cílem projektu Companions bylo vytvořit virtuální společnici, která si bude
povídat se seniory a zabaví je tak. Je to dobrý nápad a věřím, že dialogové
systémy tohoto typu mají velký smysl. V tomto případě byla doména omezena
na rozhovory týkající se rodinných fotek. Toto omezení bylo použito
pravděpodobně proto, aby se snížilo potřebné množství dat pro trénink modelů a
aby se zmenšila velikost ručně vytvářené sítě dialogu. Omezení je tedy
pochopitelné a věřím, že s větším množstvím prostředků by šlo vyvinout
univerzálnější systém, který bude dosahovat podobné kvality.

Abych mohl zhodnotit úspěch nebo neúspěch projektu, musel bych znát skutečné
cíle tohoto projektu. Pokud šlo o aplikovaný výzkum a skutečným cílem bylo
vytvořit hotový a odladěný produkt, který se bude používat k zabavení seniorů,
tak projekt skončil neúspěchem. Přeci jenom do použitelného produktu schází
mnoho.

Pokud šlo o základní výzkum, tak věřím, že projekt dopadl, zvláště po zmíněných
problémech s financováním, velmi dobře. Zdá se mi, že řešitelé projektu na to
šli správným směrem a dokázali, že architektura jejich systému funguje. Projekt
má velký smysl do budoucna a věřím, že až já budu senior, tak se podobní
společníci budou běžně používat. Spíše jsem zvědavý, jako moc mi bude vadit, že
si povídám s umělou inteligencí. Jsem přesvědčený, že takoví společníci budou v
principu fungovat podobně jako společník z projektu Companions.

Za velmi důležité považuji také to, že se takové aplikace, i když nedosahují
takových kvalit jako ty pro angličtinu, vytváří i pro češtinu. Podobné aplikace
se budou vyvíjet čím dál častěji a je dobře, že i v České republice jsou lidé s
potřebnými zkušenostmi a hlavně se zkušenostmi s češtinou.

Při praktické ukázce bohužel nefungoval spolehlivě modul rozpoznání řeči.
Nedokážu posoudit, do jaké míry to byla chyba špatné konfigurace a do jaké míry
za to mohl špatný samotný modul. Mam také podezření, že za to částečně mohl i
nekvalitní mikrofon. Kvalitní mikrofon totiž snímá pouze úzký a krátký kužel
před sebou a odfiltruje tak většinu ruchů.

Závěrem bych ještě rád dodal, že nesouhlasím s názorem, že se tento systém
vyvíjí proto, abychom mohli se seniory trávit ještě méně času než teď. Osobní
kontakt s vlastními rodiči a prarodiči bude vždy ze všeho nejdůležitější a
nevěřím tomu, že by si někdo řekl, že už nemusí se svými rodiči či prarodiči
trávit čas, protože mají náhradu v podobě virtuálního společníka. Vždy budou
chvíle, kdy budou senioři sami, například během pracovního dne, a v takových
chvílích můžou být za virtuálního společníka vděční.

\end{document}
