\documentclass[12pt,a4paper]{report}

% Babel, písmo, kódování 
\usepackage[czech]{babel}
\usepackage[utf8]{inputenc}
\usepackage{lmodern}
\usepackage[T1]{fontenc}
\usepackage[babel=true]{microtype}

\begin{document}
\section*{Automatická disambiguace podle Corpus Pattern Analysis}

Vincent Kříž a jeho tým provádějí automatickou lexikální disambiguaci na
pilotním vzorku třiceti anglických sloves. Místo obvyklých slovníkových definic
však predikují vzory slovesných užití, takzvané patterns, které vycházejí z
metody Corpus Pattern Analysis. Tato lexikografická metoda shlukuje výskyty
sloves pomocí morfosyntaktických, lexikálních a sémantických podobností svého
kontextu namísto abstraktních slovníkových definic. Těžištěm práce Vincenta
Kříže však je takzvaný Feature Engineering, tedy výběr správných rysů a metod
pro strojové učení. Kromě základních morfosyntaktických a sémantických rysů se
dále v práci zkoušejí rysy ušité na míru jednotlivým slovesům.  Procedura pro
extrakci těchto rysů je plně automatická a je založena na analýze vzorů užití
daného slovesa.

S disambiguací významů slov mám dokonce hned dvě zkušennosti ze školních
projektů.  Jako zápočtovou práci na předmět Úvod do strojového učení jsme měli
za úkol natrénovat tři klasifikátory pro slova 'hard', 'line' a 'serve'. Na
rozdíl od práce Vincenta Kříže jsem však predikoval obyčejné slovníkové
definice.  Z trénovacích dat jsem extrahoval morfosyntaktické rysy velmi
podobné těm, které použil Vincent Kříž ve svém modelu U1. Z morfologických rysů
to konkrétně byly lemma, morfologická značka a forma cílového slova a dále
lemmata a morfologické značky okolních slov.  Ze syntaktických rysů to byly
analytické funkce cílového slova a jeho rodiče. Z extrahovaných rysů jsem dále
vybral podmnožinu pomocí dopředného hladového algoritmu podobně, jako to dělal
ve své práci Vincent. Vyzkoušel jsem různé metody strojového učení a podobně
jako Vincentovi mi nejlépe vycházela metoda SVM.

Druhou zkušennost s touto problematikou jsem získal při práci na úkolu pro
předmět lingvistické zdroje dat. Ve dvojicích jsme si měli vymyslet nějaký
anotační miniprojekt, chvíli anotovat a nakonec změřit mezianotátorskou shodu.
S kolegyní jsme se rozhodli, že budeme disambiguovat významy českých sloves
'ležet', 'stát' a 'vidět'. Cohenova kappa nám pro tyto tři slovesa vyšla
popořadě 0.76, 0.81 a 0.60.  To ukazuje, že tato úloha není někdy jednoduchá
ani pro lidi a nelze dosáhnout o mnoho lepších výsledků ani automaticky. Je
však pravda, že celkem nízká mezianotátorská shoda byla způsobena vágností a
překryvem některých slovníkových definic z použitého slovníku. S metodou Corpus
Pattern Analysis, kterou použil Vincent Kříž ve své práci, nemám žádné
zkušennosti a je docela možné, že na ní mezianotátorská shoda dosahuje mnohem
lepších hodnot.

Na těchto příkladech jsem chtěl ukázat, že disambiguace slovních významů je
poměrně složitá disciplína, ale základní postupy řešení jsou vždy stejné.



\end{document}
