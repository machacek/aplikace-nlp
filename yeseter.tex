\documentclass[12pt,a4paper]{report}

% Babel, písmo, kódování 
\usepackage[czech]{babel}
\usepackage[utf8]{inputenc}
\usepackage{lmodern}
\usepackage[T1]{fontenc}
\usepackage[babel=true]{microtype}

\begin{document}
\section*{Sledování sítí a analýza sentimentu}

Sledování sítí a automatická analýza sentimentu jsou, podle mého názoru, dnes
trochu přeceňovány. Velká poptávka po těchto službách je způsobena potřebou vše
kvantifikovat, která vzniká ve všech trochu větších firmách.  Řekl bych, že
zkušený marketingový specialista se dozví mnohem více, pokud si bude ručně
číst jednotlivé komentáře uživatelů na správném místě. Užitečnost podobných
služeb vidím tedy zejména v tom, že naleznou zajímavé komentáře, které pak
někdo manuálně zpracuje. 

Přeceňování těchto služeb však nekladu za vinu firmě Yeseter, která se pouze
snaží vyhovět této poptávce a vydělat na ní. Hlavní dojem, který jsem si odnesl
z prezentace, je, že ohledně dolování dat z veřejných webových komentářů na to
jde firma Yeseter celkem dobře. 

Zdálo se mi, že se správou skriptů na dolování těchto dat, které se musí
vytvářet pro každou sledovanou stránku zvlášť, musí být velmi mnoho práce.
Kladu si tedy otázku, zda by to nešlo dělat nějak chytřeji: buď nějakou
heuristikou, která by automaticky rozpoznala uzly DOM stromu obsahující
komentáře, nebo nějakým grafickým nástrojem, který by umožnil i
nekvalifikovaným uživatelům označovat HTML elementy s komentáři.

Překvapilo mě také, že při následné analýze sentimentu jsou kličová slova stále
úspěšnější než strojové učení. Očekávál bych, že když se použijí kličová slova
jako rysy, strojové učení samo vybere signifikantní klíčová slova a bude dávat
lepší výsledky. Tento způsob navíc umožňuje snadno zapojit další rysy. 

V oblasti získávání dat ze sociálních sítí, zejména z Facebooku, na to však
firma jde poměrně složitě. Firma Yeseter musí složitě obcházet obranné
mechanismy Facebooku, které zabraňují automatickému prozkoumávání sítě.  Tyto
mechanismy se navíc, jak přiznávají sami zástupci firmy, velmi často mění.
Facebooku se nelze divit, obsah jeho sítě (který se skládá zejména ze
sociálního grafu, komentářů a takzvaných ,,lajků'') má velmi vysokou hodnotu a
Facebook chce tyto informace využívat buď sám (kvůli efektivnímu cílení
reklamy) nebo si za ně chce nechat zaplatit.  Nabízí se tedy otazka, zda by pro
firmu Yeseter nebylo výhodnější zaplatit Facebooku za autorizovaný přístup k
datům a nebojovat stále dokola s větrnými mlýny.

Moje poslední poznámka se netýká přímo firmy Yeseter ale s tématem souvisí.
Poslední dobou se na internetu objevuje velké množství komentářů, které se
tvaří jako uživatelské recenze nějakého produktu, ale ve skutečnosti jsou
vytvořeny prodejcem daného produktu. Tento jev je například velmi rozšířen v
hotelovém průmyslu. Zajímavou aplikací by bylo pomocí strojového učení
rozpoznávat tyto falešné komentáře. Taková aplikace by byla užitečná pro
samotné uživatele, provozovatele hotelových katalogů, ale i pro firmy, jako je
Yeseter, aby falšné komentáře neanalyzovaly. 




\end{document}
