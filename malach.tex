\documentclass[12pt,a4paper]{report}

% Babel, písmo, kódování 
\usepackage[czech]{babel}
\usepackage[utf8]{inputenc}
\usepackage{lmodern}
\usepackage[T1]{fontenc}
\usepackage[babel=true]{microtype}

\begin{document}
\section*{Projekt Malach}

Po uvedení filmu Shindlerův seznam natočeného v roce 1994 se režisérovi
Spielbergovi ozvaly desítky lidí, kteří hrůzy holokaustu přežili a kteří mu
chtěli vyprávět své životní příběhy. Steven Spielberg proto zřídil nadaci
Survivors of the Shoah - Visual History Foundation, která vyhledávala svědky
holokaustu a natáčela s nimi rozhovory. Natočené rozhovory pak v nadaci
zdigitalizovali a započali s ruční indexací. Velmi brzy však přišli na to, že
ruční indexování je velmi pomalé a drahé. Přes americkou agenturu NSF proto
vypsali grant na automatické zpracování rozsáhlého videoarchivu.  Cílem
projektu bylo převést automaticky řeč z archivu videonahrávek pamětníků
holokaustu na text a následně tyto výpovědi automaticky indexovat tak, aby se v
nich mohlo snadno vyhledávat.  Grant úspěšně získala John Hopkins University
společně s IBM Watson Research centrem, UFALem, Západočeskou univerzitou a s
dalšími. IBM Watson Research mělo na starosti převod nahrávek v angličtině na
text, Západočeská univerzita měla na starosti převod nahrávek ve slovanských
jazycích a UFAL měl na starosti jazykové modelování a následnou indexaci
přepsaných záznamů.

Myslím, že tento projekt má velký smysl hned na několika rovinách. Prvně je
velmi důležité to, že se vůbec zaznamenaly vzpomínky těch, co přežily hrůzy
holokaustu, a zachovaly se tak důležitá svědectví pro současné i budoucí
generace. Za druhé má velký smysl katalogizovat a indexovat videonahrávky, to
umožní nejen vědcům, ale všem zájemcům rychlý a jednoduchý přístup k tomu, co
je zajímá. Za třetí je to velmi dobrá zkušennost pro počítačové lingvisty,
kteří pracovali na něčem smysluplném a výsledek jejich práce se nadále využívá.

Na cvičení jsme si vyzkoušeli práci s videoarchivem. Samozřejmě mě zaujaly
velmi zajímavé výpovědi pamětníků a zvědavostí jsem hltal každou výpověď.
Vyhledávání v katalogu jsem použil pro nalezení výpovědí, které zmiňovaly místa
v blízkosti mého domova. To fungovalo velmi dobře a byl jsem tedy spokojen i s
technickou částí projektu. 

Celý projekt tedy hodnotím velmi kladně a přál bych si, aby jeho výsledky
přispěly k většímu rozšíření povědomí o tom, co se tenkrát dělo. Jsem totiž
přesvědčen o tom, že osobní výpovědi přímých účastníků holokaustu dají člověku
mnohem lepší představu o dějinách než učebnice dějepisu.







\end{document}
